% !TEX root = ../shrink.tex


\noind Le filtrage $f$ permet de supprimer tous les \tas qui ne sont pas des contre-exemples. Si le r�sultat est l'ensemble vide, on lui ajoute l'�l�ment initial. Finalement, on obtient

$$ \forall t \in \term, \forall d \in \N \; f(\mathtt{s\mathit{i}Depth}(t, d)) = 
	\left\{\begin{array}{cc}
		f(\mathtt{s\mathit{i}Depth}(t, d)) & \mbox{si $f(\mathtt{s\mathit{i}Depth}(t, d)) \neq \emptyset $} \\
		\{t\}& \mbox{sinon.}
	\end{array}\right.
$$

\noind Pass�e cette fonction de filtrage, on se rend compte que l'on peut am�liorer la fonction \texttt{s1} en continuant la descente de l'arbre pour trouver les \sts de m�me type que les \tas trouv�s par l'\algo\!\!. On construit alors \texttt{s1}$'$ en it�rant $f(\mathtt{s1Depth}(\cdot, d))$ sur chacun des termes de $f(\mathtt{s1Depth}(t, d))$ jusqu'� convergence.

\begin{enumerate}
	\item initialiser $E \define {t}$;
	\item initialiser un nouvel ensemble vide $E'$.
	\item pour tout �l�ment $t'$ dans $E$
	\begin{enumerate}
		\item ajouter le r�sultat de $f(\mathtt{s1Depth}(t', d))$ � $E'$.
	\end{enumerate}
	\item si $E = E'$
	\begin{enumerate}
		\item alors on s'arr�te
		\item sinon on pose $E \define E'$ et on revient � 2
	\end{enumerate}
\end{enumerate}

\noind La convergence de cet \algo ne fait pas de doute : Pour s'en convaincre, si $\left(E_n\right)_{n\in \N}$ est la suite des $E$ de l'\algo\!, on peut consid�rer la diff�rence sym�trique $ \delta_n = E_n \triangle E_{n+1}$. L'\algo est effectu� de telle sorte que d�s que un �l�ment est pr�sent dans $E_n$ et dans $E_{n+1}$ (il ne fait donc plus partie de $\delta_n$), alors il reste pr�sent pour tout $n' > n$. On peut donc se contenter de r�it�rer \texttt{s1\_aux} sur les �l�ments de $\delta_n \cap E_{n+1}$. On se rend alors compte que la taille maximale des �l�ments de $\delta_n$ d�croit strictement entre deux $\delta_n$ non vides successifs, ce qui impose qu'il existe un $n$ tel que $\delta_n$ soit vide, c'est � dire que $E_n = E_{n+1}$.\\

\noind En ce qui concerne $f \circ \mathtt{s2Depth}(\cdot, d)$, on n'a pas besoin de r�it�rer l'\algo car on travaille exclusivement sur les constructeurs du type de la racine : $$ \mathtt{s2}' = f \circ \mathtt{s2Depth}.$$





