% !TEX root = ../shrink.tex

\noind On peut maintenant exprimer l'algorithme de r�duction des contre-exemples dans son ensemble. Si $t$ est un contre-exemple :

\begin{enumerate}
\item on pose $d$ la profondeur de $t$.
\item on pose $E \define \{t\}$.
\item pour $i$ variant de $0$ � $d$
	\begin{enumerate}
	\item on pose $E' \define \emptyset$.
	\item pour tout �l�ment $t'$ dans $E$.
		\begin{enumerate}
		\item on ajoute $\mathtt{s1}'(t', i)$ � $E'$.
		\end{enumerate}
	\item on pose $E'' \define \emptyset$.
	\item pour tout �l�ment $t'$ dans $E'$.
		\begin{enumerate}
		\item on ajoute $\mathtt{s2}'(t', i)$ � $E''$.
		\end{enumerate}
	\item $E \define E''$
	\end{enumerate}
\item on retourne le \ta de $E$ de taille minimale.
\end{enumerate}