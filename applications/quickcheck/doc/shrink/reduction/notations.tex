% !TEX root = ../shrink.tex

\noind On d�finit une \textit{signature} $s \in \sig $ comme le quadruplet $(\type, \cons, \dom, \codom)$ o�
\begin{itemize}
\item $\type$ est un ensemble de \textit{types},
\item $\cons$ est un ensemble de \textit{constructeurs},
\item $\dom : \cons \rightarrow \displaystyle{\bigcup_{n \in \N}} \type^n$ donne le \textit{domaine} d'un constructeur,
\item $\codom : \cons \rightarrow \type$ donne le \textit{codomaine} d'un constructeur.\\
\end{itemize}

\noind L'\textit{arit�} d'un constructeur $c \in \cons$ est l'entier $n$ tel que $\dom(c) \in \type^n$. On peut maintenant d�finir par induction ce qu'est un \textit{\ta}.
\begin{itemize}
\item Tout constructeur d'arit� nulle est un terme.
\item Pour tout constructeur $c \in \cons$ d'arit� $n$ et les \tas $t_1, t_2, \dots, t_n$, alors $c(t_1, t_2, \dots, t_n)$ est un terme.
\end{itemize}
L'ensemble des \tas sur la signature $s$ est not� $\term_s$. Si $t = c(t1_, t_2, \dots, t_n)$, on dira que $c$ est le constructeur de $t$ et que $t_1, t_2, \dots, t_n$ sont ses \textit{\stds}et que $t_i$ est le $i$-�me \std\!\!. De m�me, un \textit{\st\!} de $t$ est soit un \std de $t$ soit un \st d'un des \stds de $t$. L'ensemble des \sts de $t$ est not� $\ST(t)$. Enfin, le \textit{chemin} $\omega_{t,t'}$ d'un \ta $t$ � l'un de ses \sts est une suite finie $u_1u_2\dots u_n \in \N^\N$ permettant de d�finir une suite de \tas $t_0t_1\dots t_n$ telle que 

$$ 
	\left\{
		\begin{array}{c}
			t_0 = t \\
			t_n = t' \\
			\forall k \in \ent{1,n}, \; t_k \mbox{ est le $u_k$-�me \st de $t_{k-1}$}
		\end{array}
	\right.
$$
\noind De par la structure d'arbre d'un terme, tout chemin entre deux \tas est unique. On peut donc d�finir la distance entre un \ta et l'un de ses \st comme la longueur du chemin les s�parant. On utilisera �galement la concat�nation de suites finies : si $\omega$ est une suite finie et $a$ une suite, alors $\omega a$ est la suite concat�n�e de $\omega$ et de $a$.\\

\noind De plus, on dira qu'un \ta $t$ \textit{d�rive} d'un type $T\in \type$ \ssi le constructeur de $t$ a $T$ pour codomaine. L'ensemble des \tas d�rivant d'un type $T$ est not� $\DERIVE(T)$. On �tendra cette notation aux termes : pour tout $t \in \term_s, \; \DERIVE(t)$ est l'ensemble des \tas d�rivant du m�me type que $t$.\\

\noind On pose �galement que la \textit{taille} d'un \ta est le nombre de constructeurs le composant, ou encore 

$$\forall t = c(t_1, t_2, \dots, t_n) \in \term_s, \; \size(t) = 1 + \sum_{i=1}^n \size(t_i). $$


\noind Ainsi, � partir d'un \ta $t \in \term_s$ faisant �chouer la validation de la formule, on cherche � construire un \ta $t'$ de plus petite taille continuant de faire �chouer la validation.
















