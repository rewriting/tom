\documentclass{article}
%\usepackage{amsmath}

\newcommand{\tom}{\textsc{Tom} }
\newcommand{\filter}{\ll}

\title{Document de travail :\\ Impl\'ementation des contraintes dans \tom}
\author{Anne-Claire Lonchamp, Pierre-Etienne Moreau}
\date\today

\begin{document}
\maketitle

\section{Syntaxe des contraintes dans \tom}

\subsection{Actuellement}

Les conditions, ou gardes, ont \'et\'e introduites dans \tom r\'ecemment.
\\

Actuellement les conditions se pr\'esentent sous la forme de fonctions (Java ou \tom) retournant un bool\'een.
Seuls des variables instanti\'ees dans le pattern du \%match, ou du pattern filtrant pr\'ec\'edent dans le cas de \%match imbriqu\'es, peuvent \^etre pass\'ees en arguments.
\\\\
$\%\:match(s)\:\{$

$p\:when\:c\rightarrow\{\dots\}$

$p\:when\:c_1$\textasciicircum\dots\textasciicircum$ c_n\rightarrow\{\dots\}$
\\$\}$

\subsection{Proposition pour la nouvelle syntaxe}
$\%\:advanced\:\{$

$(p\filter s)$\textasciicircum$c\rightarrow\{\dots\}$

$(p_1\filter s_1)$\textasciicircum\dots\textasciicircum$(p_n\filter s_n)$\textasciicircum$c_1$\textasciicircum\dots\textasciicircum$c_n\rightarrow\{\dots\}$
\\$\}$
\\\\Deux cas de figures se pr\'esentent :
\begin{itemize}
\item $s_1\dots s_n$ sont ind\'ependants. On accroche les gardes simplement et la r\'esolution se fait dans l'ordre d'\'ecriture des patterns.
\item $s_1\dots s_n$ sont li\'es. L'ordre de r\'esolution doit cette fois ci \^etre d\'etermin\'e par un graphe de d\'ependances. Pour cette raison, les cycles de d\'ependances ne sont pas permis. 
\end{itemize}

\subsubsection{M\'ecanisme}
Il convient donc en premier lieu de d\'eterminer un ordre entre les diff\'erents \'el\'ements. Notons $P_n$ l'\'el\'ement $(p_n \filter s_n)$, $L_n$ l'ensemble des variables contenues dans $p_n$ et $R_n$ l'ensemble des variables contenues dans $s_n$.\\

Nous dirons que $P_i < P_j$ lorsque que $L_i \cap R_j \ne \emptyset$. Notons qu'il ne s'agit pas d'une relation d'ordre au sens math\'ematique du terme.\\

En d\'eterminant les relations d'ordre des \'el\'ements entre eux il est possible de contruire un graphe de d\'ependances.  En effet $a < b$ peut \^etre interpr\'et\'e comme une ar\^ete de $a$ vers $b$.

Lorsque l'on recontre un cas tel que $a < b$ et $b < a$, on peut en d\'eduire l'existence d'un cycle. Il sera donc impossible de d\'eterminer un ordre de r\'esolution.\\

Si aucune relation d'ordre n'a \'et\'e d\'etect\'e, cela signifie l'ind\'ependance de l'\'el\'ement. Dans ce cas l'ordre de r\'esolution importe peu.\\

En parcourant le graphe, il est possible d'obtenir l'ordre dans lequel les diff\'erents \'el\'ements doivent \^etre trait\'es. L'ordre des arbres entre eux importe peu car ils sont ind\'ependants.

\subsubsection{R\'esolution}
Une fois l'ordre d\'etermin\'e, la r\'esolution d'un $\%advanced$ peut se ramener à une suite de \%match imbriqu\'es. En effet :
\\\\
$\%\:advanced\:\{$

$(p_1\filter s_1)$\textasciicircum\dots\textasciicircum$(p_n\filter s_n)$\textasciicircum$c_1$\textasciicircum\dots\textasciicircum$c_n\rightarrow\{\dots\}$
\\$\}$
\\\\
\'equivaut \`a :
\\\\
$\%\:match(s_1)\:\{$

$p_1\:when\:c_1\rightarrow\{\:\dots\:\%\:match(s_n)\:\{\:p_n\:when\:c_n\rightarrow\{\dots\}\:\}\:\}$
\\$\}$

\subsection{Syntaxe proche de JRule}
$((\_^*,p,\_^*)\filter s)$\textasciicircum$c\rightarrow\{\dots\}$ devient : $(p\in E)$\textasciicircum$c\rightarrow\{\dots\}$
\\

Cette notation pourrait intervenir en tant qu'alternative syntaxique, sans engendrer de modification dans le traitement.

\section{Applications}

Cette nouvelle forme de r\'esolution des contraintes est particuli\`erement in\-t\'e\-res\-sante dans le cas d'instructions \%match imbriqu\'es.

\subsection{Exemple}

Dans cet exemple, l'\'element h(x) contient la variable x instanci\'ee pr\'ec\'edement et va donc en \^etre d\'ependant.
\\\\
$\%\:advanced\:\{$

$(f(x)\filter s)$\textasciicircum$(g(y) \filter h(x))\rightarrow\{\dots\}$
\\$\}$
\\\\
L'imbrication des instructions \%match se fera donc ainsi :
\\\\
$\%\:match(s)\:\{$

$f(x)\rightarrow\{\:\%\:match(h(x))\:\{\:g(y)\rightarrow\{\dots\}\:\}\:\}$
\\$\}$

\end{document}
