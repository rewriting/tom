%%%%%%%%%%%%%%%%%%%%%%%%%%%%%%%%%%%%%%%%%%%%%%%%%%%%%%%%%%%%
\part{Tutorial}
%%%%%%%%%%%%%%%%%%%%%%%%%%%%%%%%%%%%%%%%%%%%%%%%%%%%%%%%%%%%
\chapter{Tutorial and examples}
\cutname{tutorial.html}
%%%%%%%%%%%%%%%%%%%%%%%%%%%%%%%%%%%%%%%%%%%%%%%%%%%%%%%%%%%%

On top of \TOM\ directory, you can find the tutorial directory where
all examples presnted here can be found and ran.
The difficulties encountered in this tutorial is increasing with each
section and subsection. Most of the examples are based on Java and the
corresponding ATerm library for terms representation.

\section{Peano integer}
The Peano integer formalism aims to represent integer as Zero and the
successor of an integer.
\subsection{Simple}
\TOM\ allows to define such algebraic specifications. First, we define
the sort term and the 2 defined operators: zero and suc.

\typeterm\ term \{\\
\begin{quote}
 \implement           \{ ATerm \}\\
 \getfs(t)      \{ (((ATermAppl)t).getAFun()) \}\\
 \cmpfs(t1,t2)  \{ t1 == t2 \}\\
 \gets(t, n)   \{ (((ATermAppl)t).getArgument(n)) \}\\
\end{quote}
\}\\
\\
\op\ term zero \{\\
\begin{quote}
\fsym \{ fzero \}\\
\end{quote}
\}\\
\\
\op\ term suc(term) \{\\
\begin{quote}
  \fsym \{ fsuc \}\\
\end{quote}
\}\\


\subsection{Advanced}

\section{Integer and Fibonacci}
\subsection{Simple}
\subsection{Advanced}

\section{List}

\section{Apigen and the automatic generation of term representation}

\section{Polynomial expression and \TOM}
